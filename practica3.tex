\documentclass{article}
\usepackage[utf8]{inputenc}
\usepackage[spanish]{babel}

\usepackage{tikz-dependency}

\usepackage{tipa} % herramientas para manejar símbolos fonéticos
\usepackage{tipx}

\usepackage{gb4e} % ejemplos numerados

\usepackage{qtree} % árboles sintácticos

\title{Práctica 3 (LMWeb 2016)}
\author{Lourdes Santamaría}
\date{\today}

\begin{document}

\maketitle

\section{Instrucciones}

\subsection{Objetivo}

El objetivo de esta práctica es que demuestres que eres capaz de elaborar un documento en \LaTeX\ replicando otro.

\subsection{Entrega}

En este caso tendrás que replicar y adaptar a tu caso un documento de ejemplo que yo te proporciono. Solo verás mi versión en PDF: tendrás que elaborar el código \LaTeX\ por tu cuenta y generar el documento PDF final.

\subsection{Fecha límite de entrega}

La fecha tope de entrega es el \textbf{domingo 24/04 a las 23:59}.

\subsection{Instrucciones}

\begin{itemize}

\item Utilizando la herramienta que prefieras, crea un documento nuevo vacío imitando el contenido y la estructura del modelo de práctica 3 que yo te proporciono.

\item Te recomiendo avanzar poco a poco para ir probando que tu documento compila correctamente. Es más sencillo detectar errores de este modo.

\item Cuando hayas terminado, descarga de manera local en tu ordenador la versión del código fuente y tu versión PDF final y sube ambos ficheros al cmpus virtual.

\end{itemize}

\section{Práctica}

En esta sección comienza la práctica propiamente dicha. ¡Suerte!

\subsection{Introducción del texto \LaTeX\ }

\subsection{Fonética: símbolos del IPA}

En esta sección bastará con que seas capaz de representar como mínimo la siguiente frase, adaptada a tus circustancias y a tu pronunciación particular:

\begin{exe}

\ex Hola, me llamo Lourdes, nací en Madrid y soy estudiante de lingüística y lenguas aplicadas.

La transcripción fonética del ejemplo 1 es:

\begin{IPA}

[.]

\end{IPA}

\subsection{Sintáxis: análisis de constituyentes.}

Representa los árboles sintácticos de las dos siguientes oraciones:

\begin{exe}

\ex The cat is on the nap.
\ex Ayer vi a tu padre corriendo

\end{exe}

El árbol sintáctico del ejemplo 2 es:

\Tree [.S [.NP [.DT The ] [.NN cat ] ] [.VP [.VBZ is ] [.PP [.IN on ] [.NP [.DT the ] [.NN nap ] ] ] ] ]

El ejemplo 3 es ambiguo y tiene dos análisis sintácticos, igualmente correctos:

\Tree [.O [.SV [.SA [.RB Ayer ] ] [.VBC ví ] [.SP [.IN a ] [.SN [.DTP tu ] [.NN padre ] ] ] [.VBG corriendo ] ] ] ]

\Tree [.O [.SV [.SA [.RB Ayer ] ] [.VBC ví ] [.SP [.IN a ] [.SN [.DTP tu ] [.NN padre ] [.VBG corriendo ] ] ] ] ]

\end{document}
